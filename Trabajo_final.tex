\PassOptionsToPackage{unicode=true}{hyperref} % options for packages loaded elsewhere
\PassOptionsToPackage{hyphens}{url}
\PassOptionsToPackage{dvipsnames,svgnames*,x11names*}{xcolor}
%
\documentclass[
]{article}
\usepackage{lmodern}
\usepackage{amssymb,amsmath}
\usepackage{ifxetex,ifluatex}
\ifnum 0\ifxetex 1\fi\ifluatex 1\fi=0 % if pdftex
  \usepackage[T1]{fontenc}
  \usepackage[utf8]{inputenc}
  \usepackage{textcomp} % provides euro and other symbols
\else % if luatex or xelatex
  \usepackage{unicode-math}
  \defaultfontfeatures{Scale=MatchLowercase}
  \defaultfontfeatures[\rmfamily]{Ligatures=TeX,Scale=1}
\fi
% use upquote if available, for straight quotes in verbatim environments
\IfFileExists{upquote.sty}{\usepackage{upquote}}{}
\IfFileExists{microtype.sty}{% use microtype if available
  \usepackage[]{microtype}
  \UseMicrotypeSet[protrusion]{basicmath} % disable protrusion for tt fonts
}{}
\makeatletter
\@ifundefined{KOMAClassName}{% if non-KOMA class
  \IfFileExists{parskip.sty}{%
    \usepackage{parskip}
  }{% else
    \setlength{\parindent}{0pt}
    \setlength{\parskip}{6pt plus 2pt minus 1pt}}
}{% if KOMA class
  \KOMAoptions{parskip=half}}
\makeatother
\usepackage{xcolor}
\IfFileExists{xurl.sty}{\usepackage{xurl}}{} % add URL line breaks if available
\IfFileExists{bookmark.sty}{\usepackage{bookmark}}{\usepackage{hyperref}}
\hypersetup{
  pdftitle={Análisis de la productividad industrial en Uruguay},
  pdfauthor={Mateo Barletta - Eugenia Leira},
  colorlinks=true,
  linkcolor=Maroon,
  filecolor=Maroon,
  citecolor=Blue,
  urlcolor=blue,
  breaklinks=true}
\urlstyle{same}  % don't use monospace font for urls
\usepackage[margin=1in]{geometry}
\usepackage{graphicx,grffile}
\makeatletter
\def\maxwidth{\ifdim\Gin@nat@width>\linewidth\linewidth\else\Gin@nat@width\fi}
\def\maxheight{\ifdim\Gin@nat@height>\textheight\textheight\else\Gin@nat@height\fi}
\makeatother
% Scale images if necessary, so that they will not overflow the page
% margins by default, and it is still possible to overwrite the defaults
% using explicit options in \includegraphics[width, height, ...]{}
\setkeys{Gin}{width=\maxwidth,height=\maxheight,keepaspectratio}
\setlength{\emergencystretch}{3em}  % prevent overfull lines
\providecommand{\tightlist}{%
  \setlength{\itemsep}{0pt}\setlength{\parskip}{0pt}}
\setcounter{secnumdepth}{-2}
% Redefines (sub)paragraphs to behave more like sections
\ifx\paragraph\undefined\else
  \let\oldparagraph\paragraph
  \renewcommand{\paragraph}[1]{\oldparagraph{#1}\mbox{}}
\fi
\ifx\subparagraph\undefined\else
  \let\oldsubparagraph\subparagraph
  \renewcommand{\subparagraph}[1]{\oldsubparagraph{#1}\mbox{}}
\fi

% set default figure placement to htbp
\makeatletter
\def\fps@figure{htbp}
\makeatother

\usepackage{booktabs}
\usepackage{longtable}
\usepackage{array}
\usepackage{multirow}
\usepackage{wrapfig}
\usepackage{float}
\usepackage{colortbl}
\usepackage{pdflscape}
\usepackage{tabu}
\usepackage{threeparttable}
\usepackage{threeparttablex}
\usepackage[normalem]{ulem}
\usepackage{makecell}
\usepackage{xcolor}

\title{Análisis de la productividad industrial en Uruguay}
\usepackage{etoolbox}
\makeatletter
\providecommand{\subtitle}[1]{% add subtitle to \maketitle
  \apptocmd{\@title}{\par {\large #1 \par}}{}{}
}
\makeatother
\subtitle{Métodos para la Medición de la Eficiencia y la Productividad}
\author{Mateo Barletta - Eugenia Leira}
\date{15/01/2020}

\begin{document}
\maketitle

\renewcommand{\figurename}{Gráfico}
\renewcommand{\tablename}{Tabla}

\hypertarget{introducciuxf3n}{%
\section{Introducción}\label{introducciuxf3n}}

En este trabajo nos proponemos realizar una primera aproximación a la
productividad del sector industrial de Uruguay a partir de los datos de
la Encuesta Anual de Actividad Económica (EAAE) realizada por el
Instituto Nacional de Estadística (INE). En particular este estudio
permitirá identificar cuales sectores industriales mostraron un mejor
comportamiento productivo.

La productividad total de los factores es un tema con escasa
investigación en la agenda de Uruguay pero que ocupa un rol central en
la agenda de crecimiento. La poca disponibilidad de datos, así como el
alto rezago en la publicación de los mismos hace de esta temática un
tema poco abordado por la academia nacional. Adicionalmente la gran
mayoría de los trabajos existentes se basan en técnicas de contabilidad
del crecimiento.

El sector industrial representa un interesante objeto de estudio por la
heterogeneidad que lo compone. La metodología nos permite realizar
distintas agrupaciones, por código CIIU, por tamaño de empresa entre
otras. Los datos se publican para 22 divisiones distintas lo que da
versatilidad. Debido a la disponibilidad de datos el sector industrial
es el más estudiado en los trabajos de productividad, por lo que existe
una literatura de referencia.

Para llevar a cabo el estudio se aplicará la metodología del Índice de
Malmquist para el sector industrial uruguayo. Así como también se
descompondrá la variación del cambio tecnológico (TC) y cambios en la
eficiencia técnica (TEC).

Existen dos caminos para el cálculo de la productividad. Una primera
alternativa y la más utilizada es utilizar las técnicas de contabilidad
de crecimiento. Estas técnicas permiten descomponer el crecimiento del
producto según el crecimiento de los \emph{inputs} y así obtener el
residuo de Solow - en honor al investigador Robert Solow - también
interpretado como cambio tecnológico o cambios en la productividad. Este
método ampliamente utilizado en las últimas décadas (el trabajo pionero
de Solow es de 1957) necesita asumir una función de producción del tipo
paramétrico por lo que es necesario conocer los valores fijos de los
parámetros.

Abordaremos la medición de la productividad mediante otro tipo de
metodología, este es un enfoque semi-paramétrico utilizando métodos de
frontera.

\hypertarget{antecedentes}{%
\section{Antecedentes}\label{antecedentes}}

A nivel global el sector industrial ha sido uno de los impulsores de
crecimiento y desarrollo económico. No obstante, si bien en Uruguay se
intentó incorporar un modelo que impulse la industria del país, el
modelo de industrialización por sustitución de importaciones (ISI),
debido a la dimensión del tamaño interno y la importancia del sector
agroexportador, este proceso se extinguió y no fue sostenible en el
tiempo.

A pesar de lo anterior la industria en Uruguay es uno de los rubros
generadores de empleo más importantes por lo cual se torna relevante
estudiar su desempeño. Adicionalmente si se considera a la industria uno
de los portadores de cambio técnico resulta relevante analizar cómo se
ha comportado en los últimos años.

El objetivo de este estudio es brindar una aproximación sobre el
desempeño productivo del sector industrial en Uruguay para el periodo
2012-2016 desde una perspectiva de eficiencia económica. Por esto
resulta importante relevar la literatura que estudia la productividad de
la industria en Uruguay.

La medición de la productividad no es directa ya que es posible que una
unidad productiva utilice distintas combinaciones de insumos en el
tiempo que difieran con otras unidades productivas. Existen diversos
trabajos cuyo objetivo es estudiar la productividad del sector
industrial en Uruguay.

En este sentido, \emph{Casacuberta y Gandelman (2015)} estudian el
desempeño productivo de varios sectores económicos de Uruguay,
específicamente su objetivo es identificar cual fue el efecto de la
crisis financiera que experimentó el país en el año 2002 sobre la
productividad total de los factores (\emph{tfp}) de las distintas
actividades económicas del país. Para ello, utilizan datos
proporcionados por la Encuesta Anual de Actividad Económica del INE para
el período 1997-2005 y combinan el uso de dos metodologías, por un lado,
la estimación de productividad siguiendo la metodología de
\emph{Levinsohn y Petrin (2003)}, y por otro lado computaron la
productividad total de los factores de \emph{Hsieh and Klenow (2009)}.
De su análisis concluyen que el cierre de las firmas está asociado con
niveles menores de productividad. En particular, el sector industrial
del país evidenció en promedio un detrimento de su productividad en los
años estudiados.

En otro trabajo, \emph{Carracelas, Casacuberta y Vaillant (2009)} en su
estudio sobre el desempeño sectorial discuten las posibles causantes de
la evidencia contradictoria en los estudios de la productividad total de
los factores en Uruguay. Adicionalmente analizan el desempeño productivo
de los diversos sectores de la economía del país. Este estudio utiliza
diversas fuentes de datos tales como Banco Central del Uruguay (BCU),
INE y el Instituto de Economía (IECON) a partir del método de números
índices de Törnqvist para el período 1991-2007. En particular,
encuentran ganancias de productividad en promedio para el sector
industrial en los años bajo estudio, resultados opuestos a los
encontrados por \emph{Casacuberta y Gandelman (2015)}.

De lo anterior se desprende que no se ha encontrado evidencia empírica
concluyente respecto al desempeño productivo del sector industrial en
Uruguay. Por esto es interesante analizar cuáles son los efectos
encontrados en la productividad de este rubro utilizando un método
diferente como es el Índice de Productividad de Malmquist (MPI).

\hypertarget{datos-y-selecciuxf3n-de-variables}{%
\section{Datos y selección de
variables}\label{datos-y-selecciuxf3n-de-variables}}

Los datos utilizados en este trabajo son extraídos de la Encuesta Anual
de Actividad Económica (EAAE) realizada por el INE\footnote{\url{http://www.ine.gub.uy/web/guest/industria-comercio-y-servicios}}.
Se cuenta con información entre 2008 y 2016, aunque a los efectos de
este análisis se utilizarán aquellos a partir del 2012 debido a que se
observan algunos inconvenientes en la base de datos en el año 2011.

La EAAE se desagrega a nivel de CIIU, lo que permite realizar el
análisis para los diferentes rubros de la actividad industrial, esto
significa que los datos se publican para 22 divisiones distintas lo que
da una amplica cobertura al estudio. En el Anexo se incluye una
descripción de todas las divisiones consideradas.

Para obtener una base de datos que nos permita efectuar el estudio se
deberá depurar la base para quitar outliers y deflactar las variables
por un índice de precios agregado para obtener valores constantes de las
mismas. Los valores considerados son las macrovariables calculadas por
el INE a partir de la agregación de los datos de la EAAE ya que no se
pudo obtener acceso a los microdatos por tratarse de información
confidencial bajo secreto estadístico.

En particular, se utilizará como variable output el valor bruto de
producción (VBP) ya que esta variable refleja correctamente el valor
producido y es menos volátil que el valor agregado bruto (VAB); mientras
que como inputs se utilizarán las remuneraciones (REM), el consumo de
capital fijo (CKF) y el consumo intermedio (CI), ya que son los insumos
clásicos de cualquier función de producción utilizados en la literatura.
Con estas variables se intenta aproximar al \emph{Producto},
\emph{Capital}, \emph{Trabajo} y \emph{Consumo intermedio}
respectivamente.

Como se puede ver en la siguiente tabla, todas las variables se
encuentran ampliamente correlacionadas:

\begin{table}[H]

\caption{\label{tab:unnamed-chunk-1}Matriz de correlaciones}
\centering
\begin{tabular}[t]{lrrrr}
\toprule
  & vbp & ci & ckf & rem\\
\midrule
\rowcolor{gray!6}  vbp & 1.0000 & 0.9972 & 0.8219 & 0.9597\\
ci & 0.9972 & 1.0000 & 0.8074 & 0.9597\\
\rowcolor{gray!6}  ckf & 0.8219 & 0.8074 & 1.0000 & 0.7307\\
rem & 0.9597 & 0.9597 & 0.7307 & 1.0000\\
\bottomrule
\end{tabular}
\end{table}

Las variables se encuentran en valores corrientes. Para una mejor
aproximación de la función de producción es deseable trabajar a niveles
constantes, ya que de esta manera se aislan los efectos de los precios y
se aproxima mejor el valor unitario de los insumos y el producto. Por
esta razón se decide utilizar un deflactor específico para cada
variable, como se detalla a continuación:

\begin{itemize}
\tightlist
\item
  \textbf{Producto (Y):} Índice de precios implícitos de la producción,
  calculado por el BCU.
\item
  \textbf{Capital (K):} Índice de precios implícitos de la formación
  bruta de capital, calculado por el BCU.
\item
  \textbf{Trabajo (L):} Índice medio de salarios, calculado por el INE.
\item
  \textbf{Consumo intermedio (CI):} Índice de precios implícitos de la
  producción, calculado por el BCU.
\end{itemize}

Para simplificar la deflactación se utilizan los índices agregados para
toda la industria. El deflactar el producto y el consumo intermedio por
el mismo índice de precios conlleva el supuesto implícito de que la
evolución de precios del producto y de las materias primas es la misma.

Una vez realizada la deflactación, se cuenta con las variables en
valores constantes. Como se muestra en el \emph{Gráfico 1}, las
variables en niveles tienen una alta dispersión, por lo que se aplicará
la transformación logarítmica para el tratamiento.

\includegraphics{Trabajo_final_files/figure-latex/unnamed-chunk-2-1.pdf}

Como la metodología a emplear es muy sensible a la presencia de
\emph{outliers}, también se decide quitar a las divisiones de mayor y
menor VBP. Esta es una práctica habitual en el análisis de la producción
industrial ya que algunos sectores por su importancia y peso relativo
distorsionan el análisis, como puede ser el caso de la destilación de
petróleo o producción de celulosa. También decidimos quitar los sectores
de menor peso ya que pueden no estar bien representados en la encuesta,
como es el caso de la industria del software que es una actividad más
vinculada al área de servicios. Las divisiones quitadas son las
siguientes:

\begin{table}[H]

\caption{\label{tab:unnamed-chunk-3}Divisiones no incluídas en el análisis}
\centering
\begin{tabular}[t]{ll}
\toprule
Division & Descripcion\\
\midrule
\rowcolor{gray!6}  10 & Elaboración de productos alimenticios.\\
11 y 12 & Elaboración de bebidas y elaboración de productos de tabaco\\
\rowcolor{gray!6}  17 & Fabricación de papel y de los productos de papel.\\
19 & Fabricación de coque y de productos de la refinación del petróleo.\\
\rowcolor{gray!6}  20 & Fabricación de sustancias y productos químicos.\\
\addlinespace
26 & Fabricación de los productos informáticos, electrónicos y ópticos.\\
\bottomrule
\end{tabular}
\end{table}

En el \emph{Gráfico 2} podemos observar como queda la relación entre los
\emph{inputs} y el \emph{outputs} una vez quitados los \emph{outliers}.

\includegraphics{Trabajo_final_files/figure-latex/unnamed-chunk-4-1.pdf}

En la siguiente tabla se presentan los estadísticos descriptivos de las
variables, una vez filtrada la base:

\begin{table}[H]

\caption{\label{tab:unnamed-chunk-5}Principales estadísticos descriptivos}
\centering
\begin{tabular}[t]{llrrrrrrr}
\toprule
variable & n & mean & sd & p0 & p25 & p50 & p75 & p100\\
\midrule
\rowcolor{gray!6}  y & 40 & 4.379.680.251 & 2.677.743.842 & 755.971.296 & 2.111.111.098 & 4.514.137.944 & 5.746.680.370 & 10.150.737.464\\
k & 40 & 143.156.774 & 123.808.621 & 8.966.988 & 42.095.945 & 103.692.555 & 217.861.361 & 650.302.852\\
\rowcolor{gray!6}  l & 40 & 670.557.122 & 380.360.378 & 108.760.742 & 320.780.923 & 620.062.550 & 872.187.910 & 1.765.668.588\\
ci & 40 & 2.899.503.385 & 1.949.449.891 & 379.614.733 & 1.294.200.017 & 2.803.487.762 & 4.054.703.115 & 7.351.679.158\\
\bottomrule
\end{tabular}
\end{table}

\hypertarget{metodologuxeda}{%
\section{Metodología}\label{metodologuxeda}}

Con el fin de efectuar el estudio se aplicará el Índice de Productividad
Malmquist (MPI) que examina el crecimiento de la productividad total de
los factores. Este índice se construye midiendo la distancia radial
entre los \emph{inputs} y los \emph{outputs} observados en dos períodos,
en relación con la tecnología de referencia. Es decir, evalúa cada
observación referente al grupo en dos períodos de tiempo diferentes,
generando una medida de eficiencia para la unidad bajo análisis.

Cuando la función de distancia es orientada al \emph{input},
caracterizan una tecnología por la máxima contracción proporcional
posible en el uso de los insumos, dado un nivel del producto constante.
Mientras que cuando está orientado al \emph{output} se considera la
expansión proporcional máxima del vector de producción, dado los insumos
utilizados. El MPI, a diferencia de los índices tradicionales, realiza
de forma conjunta las asignaciones entre los dos períodos de los
\emph{inputs} y los \emph{outputs}.

En este estudio se efectuará el cálculo del MPI utilizando un método
DEA. Éste permite, a través de la programación lineal, construir una
frontera no paramétrica que contiene todas las unidades eficientes y sus
posibles combinaciones. Como resultado, aquellas unidades ineficientes
quedarán por fuera de esta frontera, midiéndose la eficiencia de cada
división con respecto a dicha frontera. Cuando la función de distancia
es orientada al \emph{input}, caracterizan una tecnología por la máxima
contracción proporcional posible en el uso de los insumos, dado un nivel
del producto constante. Mientras que cuando está orientado al
\emph{output} se considera la expansión proporcional máxima del vector
de producción, dado los insumos utilizados. El MPI, a diferencia de los
índices tradicionales, realiza de forma conjunta las asignaciones entre
los dos períodos de los \emph{inputs} y los \emph{outputs}.

En este caso, se utilizará una orientación al \emph{output} ya que para
medir los cambios en la productividad de la industria buscamos maximizar
la producción. Cuando el estudio es de orientación al \emph{output}, un
valor del índice superior a la unidad representa un incremento en la
productividad en el tiempo. Es decir, cuando el sector tiene una mejora
en la productividad (entre t y t+1), entonces el resultado arrojado por
el índice será superior a 1; por el contrario, si el resultado es
inferior a la unidad se percibe un descenso en la productividad;
mientras que si el resultado es igual a la unidad es porque no se
evidenció ningún cambio en el período bajo estudio.

El cambio productivo de una división en particular (y del sector
industrial en general) entre el período t y t+1 puede determinarse a
partir de la siguiente fórmula:

FORMULA DEL INDICE

Adicionalmente, el MPI se descompone en cambios de eficiencia técnica
(TEC), correspondiente al primer componente de la expresión que prosigue
y cambio tecnológicos (TC), correspondiente al segundo componente.

FORMULA DESCOMPOSICION

Si el cambio de eficiencia es superior a la unidad, entonces hay un
efecto de catching-up en el tiempo. Si el componente del cambio
tecnológico es superior a uno, entonces estamos frente a un progreso
tecnológico en el tiempo. Por su parte, Fare et at.(1994) plantea que a
su vez, es posible descomponer los cambios de eficiencia en cambios de
eficiencia de escala (SEC) y en cambios de eficiencia técnica pura
(PETEC). Por lo tanto, el MPI orientado al output también puede
expresarse a través de la siguiente ecuación:

FORMULA FINAL

Para obtener el el índice orientado al output es necesario calcular 208
funciones de distancia \footnote{Esto es porque es necesario calcular
  cuatro funciones distancia por división, lo que resulta de 16
  divisiones (K) por año (T=5). Al utilizar la fórmula K*(3T-2) el
  resultado que arroja es 208.}. Estas funciones representan la
eficiencia técnica, por lo que permiten comparar los requerimientos
utilizados por las unidades ineficientes a partir de los productos e
insumos utilizado por las unidades eficientes.

Este enfoque supone convexidad de datos. Esta metodología además, cuenta
con dos ventajas principales, en primer lugar no es necesario
información de precios sino que utiliza datos de unidades físicas
(cantidades) para los insumos y productos lo que permite comparar las
diferentes actividades productivas dentro del sector industrial. En
segundo lugar, no se utiliza el supuesto de maximización de beneficios o
minimización de costos y el análisis no está atado a una forma funcional
ni supuestos de especificación. Sin embargo, es muy sensible a outliers
pudiendo generar ruido en las estimaciones, por dicho motivo es que
estos no fueron consideraron para el análisis, tal como se mencionó en
la sección anterior.

Por su parte, de acuerdo a \emph{Caves (1982)} el MPI no asume un tipo
específico de retornos a escala. No obstante, en los estudios realizados
por \emph{Griffel - Tatj'e y Lovell (1995)} se prueba que bajo
rendimientos que no son constantes a escala el MPI falla en identificar
las mejoras en la productividad a través de la mejora en la eficiencia
de escala y para algunas firmas. En este estudio utilizaremos retornos
constantes a escala dado que al regresar el valor bruto de producción
con el consumo intermedio, capital y trabajo, se desprende que no se
rechaza la hipótesis nula de retornos constantes a escala.

Para justificar el supuesto de rendimientos constantes a escala
utilizado se realiza un modelo lineal MCO del \emph{output} sobre los
\emph{inputs} y se testea la hipótesis nula de que la suma de los
coeficientes es igual a la unidad.

Primero estimamos el modelo en logaritmos:

\begin{verbatim}
## 
## Call:
## lm(formula = log(y) ~ log(k) + log(l) + log(ci), data = df_filtrado)
## 
## Coefficients:
## (Intercept)       log(k)       log(l)      log(ci)  
##      1.3285       0.0558       0.2952       0.6355
\end{verbatim}

Y luego realizamos el contraste de hipótesis:

\begin{verbatim}
## Linear hypothesis test
## 
## Hypothesis:
## log(k)  + log(l)  + log(ci) = 1
## 
## Model 1: restricted model
## Model 2: log(y) ~ log(k) + log(l) + log(ci)
## 
##   Res.Df   RSS Df Sum of Sq    F Pr(>F)
## 1     77 0.163                         
## 2     76 0.158  1   0.00507 2.44   0.12
\end{verbatim}

Como resultado de la prueba de hipótesis no se rechaza la hipótesis nula
de rendimientos constantes a escala, por lo que trabajaremos con este
supuesto, ello es habitual en la literatura de productividad industrial.

\hypertarget{resultados}{%
\section{Resultados}\label{resultados}}

Con la base ya depurada y las variables deflactadas realizamos el
cálculo del Índice de Malmquist para las 16 divisiones incluídas. Como
el índice mide la variación de la productividad entre dos períodos se
indica como valor de índice para un año dado a la variación respecto al
año anterior.

Comenzamos realizando el cálculo de todas las funciones de distancia
involucradas. Estas funciones se calculan para cada período tomando como
base el período anterior. También se calcula el índice punta a punta, es
decir entre 2012 y 2016. En el siguiente gráfico se muestran la frontera
de eficiencia para el primer y el último año considerados:

\includegraphics{Trabajo_final_files/figure-latex/unnamed-chunk-8-1.pdf}

Una vez calculadas las funciones de distancia, utilizando el inverso de
las mismas se obtiene el MPI. En la siguiente tabla se muestra la
variación del índice para cada año según la división considerada. En la
última columna se muestra la variación punta a punta. La última fila
muestra el resultado de toda la industria, es decir agrupando todos los
valores en valores constantes.

\begin{table}[H]

\caption{\label{tab:unnamed-chunk-9}Evolución del Índice de Malmquist, según año y división}
\centering
\begin{tabular}[t]{lrrrr|>{\bfseries}r}
\toprule
Division & 2013 & 2014 & 2015 & 2016 & Punta\\
\midrule
\rowcolor{gray!6}  13 & 0.9186 & 1.1241 & 1.0439 & 1.0201 & 1.0866\\
14 & 1.0268 & 1.0417 & 0.9396 & 1.1124 & 1.1349\\
\rowcolor{gray!6}  15 & 1.0405 & 1.1414 & 0.9793 & 0.9359 & 1.0452\\
16 & 1.1289 & 1.0560 & 0.9496 & 0.9383 & 1.0286\\
\rowcolor{gray!6}  18 & 1.0123 & 1.0267 & 1.0087 & 0.9876 & 1.0414\\
\addlinespace
21 & 1.0240 & 0.9892 & 1.0284 & 1.0638 & 1.1050\\
\rowcolor{gray!6}  22 & 1.0102 & 1.0374 & 0.9874 & 0.9854 & 1.0160\\
23 & 1.0349 & 0.9524 & 0.9769 & 0.9888 & 0.9516\\
\rowcolor{gray!6}  24 & 1.0230 & 0.9450 & 1.0194 & 0.9722 & 0.9586\\
25 & 0.9388 & 1.0366 & 1.0059 & 0.9708 & 0.9611\\
\addlinespace
\rowcolor{gray!6}  27 & 1.0528 & 0.9952 & 1.0349 & 0.9546 & 1.0626\\
28 & 0.8875 & 1.1968 & 0.7885 & 1.1266 & 0.9126\\
\rowcolor{gray!6}  29 y 30 & 1.1159 & 0.9561 & 1.0149 & 0.9158 & 1.0140\\
31 & 0.8952 & 0.9766 & 0.9490 & 0.9886 & 0.8300\\
\rowcolor{gray!6}  32 & 1.0541 & 0.9975 & 0.9997 & 1.0021 & 1.0449\\
\addlinespace
33 & 1.1000 & 0.9086 & 1.0998 & 1.0523 & 1.1295\\
\rowcolor{gray!6}  \textbf{Promedio} & \textbf{1.0165} & \textbf{1.0238} & \textbf{0.9891} & \textbf{1.0009} & \textbf{1.0202}\\
\bottomrule
\end{tabular}
\end{table}

El índice agregado arroja que en promedio hubo una ganancia de 2,02\% en
productividad, concentrada en los años 2013 y 2014, ya que en los años
2015 y 2016 se ven caídas en el MPI. Los resultados muestran que 11 de
las 16 divisiones tuvieron ganancias de productividad en promedio entre
los años 2012 y 2016. Los sectores con mayores ganancias de
productividad en el período analizado son: \emph{14 - Fabricación de
prendas de vestir} y \emph{33 - Reparación e instalación de maquinaria y
equipo}. Por otro lado los sectores de peor desempeño son: \emph{31 -
Fabricación de muebles} y \emph{28 - Fabricación de maquinaria y
equipo}. La siguiente gráfica muestra la evolución del MPI para estas
divisiones:

\includegraphics{Trabajo_final_files/figure-latex/unnamed-chunk-10-1.pdf}

\hypertarget{descomposiciuxf3n-del-uxedndice}{%
\subsection{Descomposición del
Índice}\label{descomposiciuxf3n-del-uxedndice}}

Como se indicó en el apartado metodológico el MPI puede ser descompuesto
en cambios en la eficiencia (TEC) y cambio técnico (TC). Aplicamos esta
descomposición para la viariación punta a punta y obtenemos el TEC y el
TC. Adicionalmente se descomponen el cambio en eficiencia en cambio de
eficiencia de escala (SEC) y cambio técnico puro (PTEC). Se incluye en
la última fila el promedio simple de todas las divisiones.

En la siguiente tabla se muestran los resultados de la descomposición:

\begin{table}[H]

\caption{\label{tab:unnamed-chunk-11}Descomposición del cambio técnico según división}
\centering
\begin{tabular}[t]{lrrrrr}
\toprule
Division & TFPC & TC & TEC & PTEC & SEC\\
\midrule
\rowcolor{gray!6}  13 & 1.0866 & 1.0866 & 1.0000 & 1.0000 & 1.0000\\
14 & 1.1349 & 1.0673 & 1.0634 & 1.0710 & 0.9929\\
\rowcolor{gray!6}  15 & 1.0452 & 1.0670 & 0.9796 & 1.0800 & 0.9070\\
16 & 1.0286 & 1.0482 & 0.9813 & 0.9826 & 0.9987\\
\rowcolor{gray!6}  18 & 1.0414 & 1.1052 & 0.9423 & 0.9492 & 0.9927\\
\addlinespace
21 & 1.1050 & 1.1050 & 1.0000 & 1.0000 & 1.0000\\
\rowcolor{gray!6}  22 & 1.0160 & 1.0471 & 0.9703 & 1.0000 & 0.9703\\
23 & 0.9516 & 1.0518 & 0.9047 & 0.9065 & 0.9980\\
\rowcolor{gray!6}  24 & 0.9586 & 1.0110 & 0.9482 & 0.9548 & 0.9930\\
25 & 0.9611 & 1.0854 & 0.8855 & 0.9226 & 0.9597\\
\addlinespace
\rowcolor{gray!6}  27 & 1.0626 & 1.0765 & 0.9871 & 0.9765 & 1.0109\\
28 & 0.9126 & 0.9126 & 1.0000 & 1.0000 & 1.0000\\
\rowcolor{gray!6}  29 y 30 & 1.0140 & 1.0968 & 0.9245 & 0.9292 & 0.9949\\
31 & 0.8300 & 0.9202 & 0.9019 & 0.9195 & 0.9809\\
\rowcolor{gray!6}  32 & 1.0449 & 1.0627 & 0.9833 & 1.0000 & 0.9833\\
\addlinespace
33 & 1.1295 & 1.1025 & 1.0245 & 1.0000 & 1.0245\\
\rowcolor{gray!6}  \textbf{Promedio} & \textbf{1.0202} & \textbf{1.0529} & \textbf{0.9685} & \textbf{0.9808} & \textbf{0.9879}\\
\bottomrule
\end{tabular}
\end{table}

La descomposición nos permite observar que la ganancia de productividad
promedio de 2,02\% se puede descomponer en un una ganancia de 5,29\%
cambio técnico y una pérdida de 3,15\% de cambio en la eficiencia
técnica. Es decir que los aumentos de productividad vienen dados por
mejoras promedio en la tecnología de producción y no en la eficiencia.

Cuando analizamos las divisiones de mejor desempeño vemos que la
ganancia en productividad se explica por ganancias de similar magnitud
en TEC y TC para la división \emph{14 - Fabricación de prendas de
vestir}. En el caso de \emph{33 - Reparación e instalación de maquinaria
y equipo} vemos que el aumento de la TFPC viene de la mano de un fuerte
aumento en el cambio técnico.

\hypertarget{conclusiones}{%
\section{Conclusiones}\label{conclusiones}}

En este trabajo pudimos obtener mediciones de la productividad total de
los factores para el sector industrial para el período 2012-2016.
Encontramos que existieron ganancias de productividad en promedio para
todos los años salvo el 2015, pero con un desempeño sectorial
hetogéneno, habiendo sectores ganancias netas de productividad así como
otros con pérdidas.

La adopción metodológica del índice de Malmquist representa una
innovación en la literatura existente ya que solamente se había
utilizado métodos paramétricos para la estimación de la productividad
industrial en Uruguay. Los resultados arrojan una tasa de crecimiento de
la productividad promedio de 2,02\% para el período estudiado, lo cual
es un resultado dentro del margen de lo esperado.

A su vez la aplicación del MPI permitió descomponer el crecimiento de la
productividad en cambio tecnológico y ganancias en la eficiencia,
obteniendo que las ganancias de productividad son resultado de cambio
técnico.

\hypertarget{referencias-nibliogruxe1ficas}{%
\section{Referencias
nibliográficas}\label{referencias-nibliogruxe1ficas}}

\begin{itemize}
\item
  {[}1{]} Casacuberta, C.; Gandelman N. (2015) ``Productivity, exit and
  crisis in Uruguayan manufacturing and services sectors''. Developing
  Economies, Vol. 53(1), pp 27-43.
\item
  {[}2{]} Carracelas, G.; Casacuberta C. y Vaillant, M. (2009)
  ``Productividad total de factores: desempeño sectorial heterogéneo''.
  Facultad de Ciencias Sociales, UdelaR.
\item
  {[}3{]} Eslava, M.; Hurtado, B.; Salas, N., et al.~(2018) ``Microdatos
  para el estudio de la productividad en América Latina''. Banco de
  Desarrollo para América Latina (CAF).
\item
  {[}4{]} Lara, M. (2011) ``Desempeño de productividad sectoriales de la
  industria manufacturera uruguaya (1970-2000) en una perspectiva
  comparada''. Facultad de Ciencias Sociales, UdelaR.
\item
  {[}5{]} Martínez-Damián, M.; Brambila-Paz, J. y García-Mata, R. (2013)
  ``Índice de malmquist y productividad estatal en México''.
  Agricultura, sociedad y desarrollo, 10(3), 359-369.
\item
  {[}6{]} Presto, G. (2016) ``Efectos de la productividad y rentabilidad
  en el desempeño y supervivencia de las empresas manufactureras
  uruguayas''. Facultad de Ciencias Sociales, UdelaR.
\end{itemize}

\hypertarget{anexo}{%
\section{Anexo}\label{anexo}}

\begin{table}[H]

\caption{\label{tab:unnamed-chunk-12}Divisiones incluídas en el análisis}
\centering
\resizebox{\linewidth}{!}{
\begin{tabular}[t]{ll}
\toprule
Division & Descripcion\\
\midrule
\rowcolor{gray!6}  10 & Elaboración de productos alimenticios.\\
11 y 12 & Elaboración de bebidas y elaboración de productos de tabaco\\
\rowcolor{gray!6}  13 & Fabricación de productos textiles.\\
14 & Fabricación de prendas de vestir.\\
\rowcolor{gray!6}  15 & Fabricación de cueros y productos conexos.\\
\addlinespace
16 & Producción de madera y fabricación de productos de madera y corcho, excepto muebles\\
\rowcolor{gray!6}  17 & Fabricación de papel y de los productos de papel.\\
18 & Actividades de impresión y reproducción de grabaciones.\\
\rowcolor{gray!6}  19 & Fabricación de coque y de productos de la refinación del petróleo.\\
20 & Fabricación de sustancias y productos químicos.\\
\addlinespace
\rowcolor{gray!6}  21 & Fabricación de productos farmacéuticos, sustancias químicas medicinales y de produc\\
22 & Fabricación de productos de caucho y plástico.\\
\rowcolor{gray!6}  23 & Fabricación de otros productos minerales no metálicos.\\
24 & Fabricación de metales comunes.\\
\rowcolor{gray!6}  25 & Fabricación de productos derivados del metal, excepto maquinaria y equipo.\\
\addlinespace
26 & Fabricación de los productos informáticos, electrónicos y ópticos.\\
\rowcolor{gray!6}  27 & Fabricación de equipo eléctrico.\\
28 & Fabricación de maquinaria y equipo n.c.p.\\
\rowcolor{gray!6}  29 y 30 & Fabricación de vehículos automotores, remolques y semirremolques. Fabricación de ot\\
31 & Fabricación de muebles.\\
\addlinespace
\rowcolor{gray!6}  32 & Otras industrias manufactureras.\\
33 & Reparación e instalación de la maquinaria y equipo.\\
\bottomrule
\end{tabular}}
\end{table}

\end{document}
